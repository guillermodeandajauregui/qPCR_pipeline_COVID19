\documentclass{article}
\usepackage[$geometry$]{geometry}
\usepackage{booktabs}
\usepackage{longtable}
\usepackage{array}
\usepackage{multirow}
\usepackage{wrapfig}
\usepackage{float}
\usepackage{colortbl}
\usepackage{pdflscape}
\usepackage{tabu}
\usepackage{threeparttable}
\usepackage{threeparttablex}
\usepackage[normalem]{ulem}
\usepackage{makecell}
\usepackage{xcolor}
\usepackage[spanish]{babel}
\usepackage[utf8]{inputenc}
\usepackage[export]{adjustbox}
$if(graphics)$
\usepackage{graphicx}
\makeatletter
\def\maxwidth{\ifdim\Gin@nat@width>\linewidth\linewidth
\else\Gin@nat@width\fi}
\makeatother
\let\Oldincludegraphics\includegraphics
\renewcommand{\includegraphics}[1]{\Oldincludegraphics[width=\maxwidth]{#1}}
$endif$



\begin{document}  
\thispagestyle{plain} 
$if(graphics)$$if(logo)$\Oldincludegraphics[width=$if(width)$$width$$else$0.3$endif$\textwidth, $if(logoposition)$$logoposition$$else$right$endif$]{$logo$}$else$$endif$$endif$


\begin{center} 
{\fontfamily{cmss}\selectfont
\Large \bfseries{\scshape{Unidad de Secuenciación del Instituto Nacional de Medicina Genómica}}
}
\end{center}

\begin{flushright} 
Ciudad de México a \today
\end{flushright}

\begin{center} 
{\fontfamily{cmss}\selectfont
\large \bfseries{\scshape{Reporte Técnico de Laboratorio}}\\
}
\end{center}
\hrule
\vspace{2mm}

Estimado Dr(a): $if(receiber)$\texttt{$receiber$}$else$TBD$endif$


Encuentre a continuación el reporte de las muestras entregadas por su institución:\\

Número de muestras entregadas: $if(allsamples)$\texttt{$allsamples$}$else$TBD$endif$\\
Número de muestras procesadas: $if(nsamples)$\texttt{$nsamples$}$else$TBD$endif$\\
Fecha de procesamiento de las muestras: $if(processingdate)$\texttt{$processingdate$}$else$TBD$endif$\\
Institución solicitante: $if(id)$\texttt{$id$}$else$TBD$endif$\\  

\medskip

\textbf{Resumen de resultados:}

\medskip


Se procesaron las muestras descritas en la(s) tabla(s) adjunta (s), para la detección de material genético del SARS-CoV-2 (COVID-19) utilizando dos sondas (N1, N2) específicas para la detección de la nucleocápside del virus y utilizando la técnica de reacción en cadena de la polimerasa con retrotranscriptasa reversa en tiempo real (qRT-PCR).\\

\medskip

Encontrará los resultados en la tabla(s) adjunta (s) a este documento.\\

\smallskip

Le enviamos un cordial saludo y estamos atentos a sus dudas y/o comentarios.\\

\medskip

\vspace*{\fill}
\begin{center}
A T E N T A M E N T E
\end{center}

\vspace{2cm}
\begin{center}
\begin{tabular}{m{8cm} p{1cm} m{8cm}}
~ & ~ & ~ \\
~ & ~ & ~ \\
\cmidrule(r){1-1}\cmidrule(l){3-3}
M. en C. Alfredo Mendoza Vargas & ~ & Dr. Juan Pablo Reyes Grajeda \\
\end{tabular}

\footnotesize
\begin{tabular}{m{8cm} p{1cm} m{8cm}}
Responsable del Área Operativa & ~ & Subdirector de Desarrollo de Aplicaciones Clínicas\\
Tel. 55 5350 1972 & ~ & Responsable Sanitario ante COFEPRIS\\
mail: amendoza@inmegen.gob.mx & ~ & Tel. 55 5350 1900 ext. 1192\\
~ & ~ & mail: jreyes@inmegen.gob.mx \\
\end{tabular}
\end{center}

\clearpage

$body$


\vspace*{\fill}
{\footnotesize Las muestras \textbf{no} fueron tomadas por el INMEGEN. Fueron procesadas por el personal $brigada$ y validados por el(los) analista(s) $analista$. \par}
\end{document}

