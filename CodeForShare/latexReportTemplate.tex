\documentclass[$if(fontsize)$$fontsize$,$endif$$if(papersize)$$papersize$,$endif$$for(classoption)$$classoption$$sep$,$endfor$table]{article}
\usepackage{xcolor}
\usepackage[$geometry$]{geometry}
\newcommand*{\authorfont}{\fontfamily{cmss}\selectfont}
$if(fontfamily)$
\usepackage[$fontfamilyoptions$]{$fontfamily$}
$else$
\usepackage[T1]{fontenc}
$endif$

\usepackage{abstract}
\renewcommand{\abstractname}{}    % clear the title
\renewcommand{\absnamepos}{empty} % originally center
\newcommand{\blankline}{\quad\pagebreak[2]}

\providecommand{\tightlist}{%
  \setlength{\itemsep}{0pt}\setlength{\parskip}{0pt}} 
\usepackage{longtable,booktabs}

\usepackage{parskip}
\usepackage{titlesec}
\titlespacing\section{0pt}{12pt plus 4pt minus 2pt}{6pt plus 2pt minus 2pt}
\titlespacing\subsection{0pt}{12pt plus 4pt minus 2pt}{6pt plus 2pt minus 2pt}

\titleformat*{\subsubsection}{\normalsize\itshape}

\usepackage{titling}
\setlength{\droptitle}{-.25cm}

%\setlength{\parindent}{0pt}
%\setlength{\parskip}{6pt plus 2pt minus 1pt}
%\setlength{\emergencystretch}{3em}  % prevent overfull lines 

\usepackage[$if(fontenc)$$fontenc$$else$T1$endif$]{fontenc}
\usepackage[utf8]{inputenc}
\usepackage[spanish]{babel}
\usepackage{fancyhdr}
\pagestyle{fancy}
\usepackage{lastpage}
\renewcommand{\headrulewidth}{0.3pt}
\renewcommand{\footrulewidth}{0.0pt} 
\lhead{}
\chead{}
\rhead{\footnotesize $title$ -- $date$}
\lfoot{}
\cfoot{\small \thepage/\pageref*{LastPage}}
\rfoot{}

\fancypagestyle{firststyle}
{
\renewcommand{\headrulewidth}{0pt}%
   \fancyhf{}
   \fancyfoot[C]{\small \thepage/\pageref*{LastPage}}
}

%\def\labelitemi{--}
%\usepackage{enumitem}
%\setitemize[0]{leftmargin=25pt}
%\setenumerate[0]{leftmargin=25pt}




\makeatletter
\@ifpackageloaded{hyperref}{}{%
\ifxetex
  \usepackage[setpagesize=false, % page size defined by xetex
              unicode=false, % unicode breaks when used with xetex
              xetex]{hyperref}
\else
  \usepackage[unicode=true]{hyperref}
\fi
}
% \@ifpackageloaded{color}{
%     \PassOptionsToPackage{usenames,dvipsnames}{color}
% }{%
%     \usepackage[usenames,dvipsnames]{color}
% }
\makeatother
\hypersetup{breaklinks=true,
            bookmarks=true,
            pdfauthor={$for(author)$$author.name$ ($author.affiliation$)$sep$ and $endfor$},
             pdfkeywords = {$if(keywords)$$keywords$$endif$},  
            pdftitle={$title$$if(subtitle)$: $subtitle$$endif$},
            colorlinks=true,
            citecolor=$if(citecolor)$$citecolor$$else$blue$endif$,
            urlcolor=$if(urlcolor)$$urlcolor$$else$blue$endif$,
            linkcolor=$if(linkcolor)$$linkcolor$$else$magenta$endif$,
            pdfborder={0 0 0}}
\urlstyle{same}  % don't use monospace font for urls


\setcounter{secnumdepth}{0}

$if(listings)$
\usepackage{listings}
$endif$
$if(lhs)$
\lstnewenvironment{code}{\lstset{language=r,basicstyle=\small\ttfamily}}{}
$endif$
$if(highlighting-macros)$
$highlighting-macros$
$endif$
$if(verbatim-in-note)$
\usepackage{fancyvrb}
$endif$
$if(tables)$
\usepackage{longtable}
$endif$

$if(graphics)$
\usepackage{graphicx}
% We will generate all images so they have a width \maxwidth. This means
% that they will get their normal width if they fit onto the page, but
% are scaled down if they would overflow the margins.
\makeatletter
\def\maxwidth{\ifdim\Gin@nat@width>\linewidth\linewidth
\else\Gin@nat@width\fi}
\makeatother
\let\Oldincludegraphics\includegraphics
\renewcommand{\includegraphics}[1]{\Oldincludegraphics[width=\maxwidth]{#1}}
$endif$


$if(natbib)$
\usepackage{natbib}
\bibliographystyle{$if(biblio-style)$$biblio-style$$else$plainnat$endif$}
$endif$
$if(biblatex)$
\usepackage$if(biblio-style)$[style=$biblio-style$]$endif${biblatex}
$if(biblatexoptions)$\ExecuteBibliographyOptions{$for(biblatexoptions)$$biblatexoptions$$sep$,$endfor$}$endif$
$for(bibliography)$
\addbibresource{$bibliography$}
$endfor$
$endif$
$if(listings)$
\usepackage{listings}
$endif$

\usepackage{setspace}

$if(graphics)$
\usepackage[export]{adjustbox}
$endif$


\usepackage{float}

\begin{document}  

\thispagestyle{plain} 


$if(graphics)$$if(logo)$\Oldincludegraphics[width=$if(width)$$width$$else$0.3$endif$\textwidth, $if(logoposition)$$logoposition$$else$right$endif$]{$logo$}$else$$endif$$endif$


\begin{center} 
{\fontfamily{cmss}\selectfont
\Large \bfseries{\scshape{Unidad de Secuenciación del Instituto Nacional de Medicina Genómica}}
}
\end{center}

\begin{flushright} 
Ciudad de México a \today
\end{flushright}

\begin{center} 
{\fontfamily{cmss}\selectfont
\large \bfseries{\scshape{Reporte Técnico de Laboratorio}}\\
}
\end{center}
\hrule
\vspace{2mm}

Estimado Dr(a): $if(receiber)$\texttt{$receiber$}$else$A quien corresponda$endif$


Encuentre a continuación el reporte de las muestras entregadas por su institución:\\

Número de muestras entregadas: $if(allsamples)$\texttt{$allsamples$}$else$TBD$endif$\\
Número de muestras procesadas: $if(nsamples)$\texttt{$nsamples$}$else$TBD$endif$\\
Fecha de toma de muestra: $if(sampling)$\texttt{$sampling$}$else$TBD$endif$\\
Fecha de procesamiento de las muestras: $if(processingdate)$\texttt{$processingdate$}$else$TBD$endif$\\
Institución solicitante: $if(id)$\texttt{$id$}$else$TBD$endif$\\  

\medskip

\textbf{Resumen de resultados:}

\medskip


Se procesaron las muestras de $if(method)$\texttt{$method$}$else$TBD$endif$, descritas en la(s) tabla(s) adjunta (s), para la detección de material genético del SARS-CoV-2 (COVID-19) utilizando dos sondas (N1, N2) específicas para la detección de la nucleocápside del virus y utilizando la técnica de reacción en cadena de la polimerasa con retrotranscriptasa reversa en tiempo real (qRT-PCR).\\

\medskip

Encontrará los resultados en la tabla(s) adjunta (s) a este documento.\\

\smallskip

Le enviamos un cordial saludo y estamos atentos a sus dudas y/o comentarios.\\

\medskip

\vspace*{\fill}
\begin{center}
A T E N T A M E N T E
\end{center}
\vspace{2cm}
\begin{tabular}{c p{2cm} c}
~ & ~ & ~ \\
~ & ~ & ~ \\
\cmidrule(r){1-1}\cmidrule(l){3-3}
M. en C. Alfredo Mendoza Vargas & ~ & Dr. Juan Pablo Reyes Grajeda \\
Responsable del Área Operativa & ~ & Subdirector de Desarrollo de Aplicaciones Clínicas\\
Tel. 55 5350 1972 & ~ & Responsable Sanitario ante COFEPRIS\\
mail: amendoza@inmegen.gob.mx & ~ & Tel. 55 5350 1900 ext. 1192\\
~ & ~ & mail: jreyes@inmegen.gob.mx \\
\end{tabular}

\newpage

$body$


\vspace*{\fill}
Las muestras \textbf{no} fueron tomadas por el INMEGEN. Fueron procesadas por el personal $brigada$ y validados por el(los) analista(s) $analista$.
\end{document}

